% Options for packages loaded elsewhere
\PassOptionsToPackage{unicode}{hyperref}
\PassOptionsToPackage{hyphens}{url}
%
\documentclass[
]{article}
\usepackage{lmodern}
\usepackage{amssymb,amsmath}
\usepackage{ifxetex,ifluatex}
\ifnum 0\ifxetex 1\fi\ifluatex 1\fi=0 % if pdftex
  \usepackage[T1]{fontenc}
  \usepackage[utf8]{inputenc}
  \usepackage{textcomp} % provide euro and other symbols
\else % if luatex or xetex
  \usepackage{unicode-math}
  \defaultfontfeatures{Scale=MatchLowercase}
  \defaultfontfeatures[\rmfamily]{Ligatures=TeX,Scale=1}
\fi
% Use upquote if available, for straight quotes in verbatim environments
\IfFileExists{upquote.sty}{\usepackage{upquote}}{}
\IfFileExists{microtype.sty}{% use microtype if available
  \usepackage[]{microtype}
  \UseMicrotypeSet[protrusion]{basicmath} % disable protrusion for tt fonts
}{}
\makeatletter
\@ifundefined{KOMAClassName}{% if non-KOMA class
  \IfFileExists{parskip.sty}{%
    \usepackage{parskip}
  }{% else
    \setlength{\parindent}{0pt}
    \setlength{\parskip}{6pt plus 2pt minus 1pt}}
}{% if KOMA class
  \KOMAoptions{parskip=half}}
\makeatother
\usepackage{xcolor}
\IfFileExists{xurl.sty}{\usepackage{xurl}}{} % add URL line breaks if available
\IfFileExists{bookmark.sty}{\usepackage{bookmark}}{\usepackage{hyperref}}
\hypersetup{
  pdftitle={R Forward MTCARS},
  pdfauthor={Yosef Guevara Salamanca},
  hidelinks,
  pdfcreator={LaTeX via pandoc}}
\urlstyle{same} % disable monospaced font for URLs
\usepackage[margin=1in]{geometry}
\usepackage{color}
\usepackage{fancyvrb}
\newcommand{\VerbBar}{|}
\newcommand{\VERB}{\Verb[commandchars=\\\{\}]}
\DefineVerbatimEnvironment{Highlighting}{Verbatim}{commandchars=\\\{\}}
% Add ',fontsize=\small' for more characters per line
\usepackage{framed}
\definecolor{shadecolor}{RGB}{248,248,248}
\newenvironment{Shaded}{\begin{snugshade}}{\end{snugshade}}
\newcommand{\AlertTok}[1]{\textcolor[rgb]{0.94,0.16,0.16}{#1}}
\newcommand{\AnnotationTok}[1]{\textcolor[rgb]{0.56,0.35,0.01}{\textbf{\textit{#1}}}}
\newcommand{\AttributeTok}[1]{\textcolor[rgb]{0.77,0.63,0.00}{#1}}
\newcommand{\BaseNTok}[1]{\textcolor[rgb]{0.00,0.00,0.81}{#1}}
\newcommand{\BuiltInTok}[1]{#1}
\newcommand{\CharTok}[1]{\textcolor[rgb]{0.31,0.60,0.02}{#1}}
\newcommand{\CommentTok}[1]{\textcolor[rgb]{0.56,0.35,0.01}{\textit{#1}}}
\newcommand{\CommentVarTok}[1]{\textcolor[rgb]{0.56,0.35,0.01}{\textbf{\textit{#1}}}}
\newcommand{\ConstantTok}[1]{\textcolor[rgb]{0.00,0.00,0.00}{#1}}
\newcommand{\ControlFlowTok}[1]{\textcolor[rgb]{0.13,0.29,0.53}{\textbf{#1}}}
\newcommand{\DataTypeTok}[1]{\textcolor[rgb]{0.13,0.29,0.53}{#1}}
\newcommand{\DecValTok}[1]{\textcolor[rgb]{0.00,0.00,0.81}{#1}}
\newcommand{\DocumentationTok}[1]{\textcolor[rgb]{0.56,0.35,0.01}{\textbf{\textit{#1}}}}
\newcommand{\ErrorTok}[1]{\textcolor[rgb]{0.64,0.00,0.00}{\textbf{#1}}}
\newcommand{\ExtensionTok}[1]{#1}
\newcommand{\FloatTok}[1]{\textcolor[rgb]{0.00,0.00,0.81}{#1}}
\newcommand{\FunctionTok}[1]{\textcolor[rgb]{0.00,0.00,0.00}{#1}}
\newcommand{\ImportTok}[1]{#1}
\newcommand{\InformationTok}[1]{\textcolor[rgb]{0.56,0.35,0.01}{\textbf{\textit{#1}}}}
\newcommand{\KeywordTok}[1]{\textcolor[rgb]{0.13,0.29,0.53}{\textbf{#1}}}
\newcommand{\NormalTok}[1]{#1}
\newcommand{\OperatorTok}[1]{\textcolor[rgb]{0.81,0.36,0.00}{\textbf{#1}}}
\newcommand{\OtherTok}[1]{\textcolor[rgb]{0.56,0.35,0.01}{#1}}
\newcommand{\PreprocessorTok}[1]{\textcolor[rgb]{0.56,0.35,0.01}{\textit{#1}}}
\newcommand{\RegionMarkerTok}[1]{#1}
\newcommand{\SpecialCharTok}[1]{\textcolor[rgb]{0.00,0.00,0.00}{#1}}
\newcommand{\SpecialStringTok}[1]{\textcolor[rgb]{0.31,0.60,0.02}{#1}}
\newcommand{\StringTok}[1]{\textcolor[rgb]{0.31,0.60,0.02}{#1}}
\newcommand{\VariableTok}[1]{\textcolor[rgb]{0.00,0.00,0.00}{#1}}
\newcommand{\VerbatimStringTok}[1]{\textcolor[rgb]{0.31,0.60,0.02}{#1}}
\newcommand{\WarningTok}[1]{\textcolor[rgb]{0.56,0.35,0.01}{\textbf{\textit{#1}}}}
\usepackage{graphicx,grffile}
\makeatletter
\def\maxwidth{\ifdim\Gin@nat@width>\linewidth\linewidth\else\Gin@nat@width\fi}
\def\maxheight{\ifdim\Gin@nat@height>\textheight\textheight\else\Gin@nat@height\fi}
\makeatother
% Scale images if necessary, so that they will not overflow the page
% margins by default, and it is still possible to overwrite the defaults
% using explicit options in \includegraphics[width, height, ...]{}
\setkeys{Gin}{width=\maxwidth,height=\maxheight,keepaspectratio}
% Set default figure placement to htbp
\makeatletter
\def\fps@figure{htbp}
\makeatother
\setlength{\emergencystretch}{3em} % prevent overfull lines
\providecommand{\tightlist}{%
  \setlength{\itemsep}{0pt}\setlength{\parskip}{0pt}}
\setcounter{secnumdepth}{-\maxdimen} % remove section numbering

\title{R Forward MTCARS}
\author{Yosef Guevara Salamanca}
\date{26/11/2020}

\begin{document}
\maketitle

Librerias a usar

\begin{Shaded}
\begin{Highlighting}[]
\KeywordTok{library}\NormalTok{(MASS)}
\KeywordTok{library}\NormalTok{(lmtest)}
\end{Highlighting}
\end{Shaded}

\begin{verbatim}
## Loading required package: zoo
\end{verbatim}

\begin{verbatim}
## 
## Attaching package: 'zoo'
\end{verbatim}

\begin{verbatim}
## The following objects are masked from 'package:base':
## 
##     as.Date, as.Date.numeric
\end{verbatim}

\begin{Shaded}
\begin{Highlighting}[]
\KeywordTok{library}\NormalTok{(nortest)}
\KeywordTok{library}\NormalTok{(leaps)}
\KeywordTok{require}\NormalTok{(stats)}
\KeywordTok{require}\NormalTok{(graphics)}
\end{Highlighting}
\end{Shaded}

carga de los datos

\begin{Shaded}
\begin{Highlighting}[]
\NormalTok{STATE.econo <-}\StringTok{ }\KeywordTok{read.csv}\NormalTok{(}\StringTok{"C:/Users/yosef/OneDrive/Documents/Esp_Estadistica/Applied-regression-methods-/MLR/Dataframes/stateed.csv"}\NormalTok{, }\DataTypeTok{header=}\NormalTok{T,}\DataTypeTok{sep=}\StringTok{";"}\NormalTok{)}
\KeywordTok{attach}\NormalTok{(STATE.econo)}
\KeywordTok{head}\NormalTok{(STATE.econo)}
\end{Highlighting}
\end{Shaded}

\begin{verbatim}
##        STATE ELDERLY  INCOME POPULATION STATEAID SCHOOLAGE
## 1    Alabama    12.9  108322       4089     2075      19.0
## 2     Alaska     4.2   25188        570     4757      21.5
## 3    Arizona    13.2  125419       3750     1767      18.7
## 4   Arkansas    14.9   60486       2372     2077      19.2
## 5 California    10.5 1112819      30380     3288      18.1
## 6   Colorado    10.1  115743       3377     1811      18.5
\end{verbatim}

Se crea el modelo completo para iniciar su interpretacion

\begin{Shaded}
\begin{Highlighting}[]
\NormalTok{STATE.full.fit<-}\KeywordTok{lm}\NormalTok{(STATEAID }\OperatorTok{~}\StringTok{ }\NormalTok{ELDERLY }\OperatorTok{+}\StringTok{ }\NormalTok{INCOME}\OperatorTok{+}\NormalTok{POPULATION }\OperatorTok{+}\StringTok{ }\NormalTok{SCHOOLAGE, }\DataTypeTok{data =}\NormalTok{ STATE.econo)}
\KeywordTok{summary}\NormalTok{(STATE.full.fit)}
\end{Highlighting}
\end{Shaded}

\begin{verbatim}
## 
## Call:
## lm(formula = STATEAID ~ ELDERLY + INCOME + POPULATION + SCHOOLAGE, 
##     data = STATE.econo)
## 
## Residuals:
##     Min      1Q  Median      3Q     Max 
## -2143.8  -452.7   -38.5   459.2  1681.1 
## 
## Coefficients:
##               Estimate Std. Error t value Pr(>|t|)   
## (Intercept)  6.037e+03  1.806e+03   3.342  0.00168 **
## ELDERLY     -1.406e+02  5.944e+01  -2.366  0.02235 * 
## INCOME       2.460e-03  4.799e-03   0.513  0.61071   
## POPULATION  -7.764e-02  1.674e-01  -0.464  0.64499   
## SCHOOLAGE   -1.068e+02  6.903e+01  -1.548  0.12868   
## ---
## Signif. codes:  0 '***' 0.001 '**' 0.01 '*' 0.05 '.' 0.1 ' ' 1
## 
## Residual standard error: 767 on 45 degrees of freedom
## Multiple R-squared:  0.1573, Adjusted R-squared:  0.08244 
## F-statistic: 2.101 on 4 and 45 DF,  p-value: 0.09643
\end{verbatim}

Como analisis previo a los calculos de los modelos propuesto de
regresion mediante el summary podemos ver que el modelo Full no es el
mas apto para explicar a STATEAID pues el p.value de INCome, POPULATION
Y SCHOOLAGE es no son signiticativamente distintos de 0

\hypertarget{lo-siguiente-es-capturar-el-conjunto-de-variables-regresoras}{%
\paragraph{Lo siguiente es capturar el conjunto de variables
regresoras}\label{lo-siguiente-es-capturar-el-conjunto-de-variables-regresoras}}

\begin{Shaded}
\begin{Highlighting}[]
\NormalTok{elegir<-}\KeywordTok{regsubsets}\NormalTok{(STATEAID }\OperatorTok{~}\StringTok{ }\NormalTok{ELDERLY }\OperatorTok{+}\StringTok{ }\NormalTok{INCOME}\OperatorTok{+}\NormalTok{POPULATION }\OperatorTok{+}\StringTok{ }\NormalTok{SCHOOLAGE, }\DataTypeTok{data =}\NormalTok{ STATE.econo)}
\KeywordTok{summary}\NormalTok{(elegir)}
\end{Highlighting}
\end{Shaded}

\begin{verbatim}
## Subset selection object
## Call: regsubsets.formula(STATEAID ~ ELDERLY + INCOME + POPULATION + 
##     SCHOOLAGE, data = STATE.econo)
## 4 Variables  (and intercept)
##            Forced in Forced out
## ELDERLY        FALSE      FALSE
## INCOME         FALSE      FALSE
## POPULATION     FALSE      FALSE
## SCHOOLAGE      FALSE      FALSE
## 1 subsets of each size up to 4
## Selection Algorithm: exhaustive
##          ELDERLY INCOME POPULATION SCHOOLAGE
## 1  ( 1 ) "*"     " "    " "        " "      
## 2  ( 1 ) "*"     " "    " "        "*"      
## 3  ( 1 ) "*"     "*"    " "        "*"      
## 4  ( 1 ) "*"     "*"    "*"        "*"
\end{verbatim}

Se grafican cada uno de los modelos entregados por regsubsets

\begin{Shaded}
\begin{Highlighting}[]
\KeywordTok{par}\NormalTok{(}\DataTypeTok{mfrow=}\KeywordTok{c}\NormalTok{(}\DecValTok{1}\NormalTok{,}\DecValTok{2}\NormalTok{))}
\KeywordTok{plot}\NormalTok{(elegir,}\DataTypeTok{scale=}\StringTok{"r2"}\NormalTok{, }\DataTypeTok{main=}\StringTok{"R^2, Datos STATE.Schmidt)."}\NormalTok{)}
\KeywordTok{plot}\NormalTok{(elegir,}\DataTypeTok{scale=}\StringTok{"adjr2"}\NormalTok{, }\DataTypeTok{main=}\StringTok{"(R^2.ajust, Datos STATE.Schmidt)."}\NormalTok{)}
\end{Highlighting}
\end{Shaded}

\includegraphics{1_punto_files/figure-latex/unnamed-chunk-5-1.pdf}

\begin{Shaded}
\begin{Highlighting}[]
\KeywordTok{par}\NormalTok{(}\DataTypeTok{mfrow=}\KeywordTok{c}\NormalTok{(}\DecValTok{1}\NormalTok{,}\DecValTok{2}\NormalTok{))}
\KeywordTok{plot}\NormalTok{(elegir, }\DataTypeTok{scale=}\StringTok{"Cp"}\NormalTok{, }\DataTypeTok{main=}\StringTok{"(Cp, Datos STATE.Schmidt)."}\NormalTok{)}
\KeywordTok{plot}\NormalTok{(elegir,}\DataTypeTok{scale=}\StringTok{"bic"}\NormalTok{, }\DataTypeTok{main=}\StringTok{"(BIC, Datos STATE.Schmidt)."}\NormalTok{)}
\end{Highlighting}
\end{Shaded}

\includegraphics{1_punto_files/figure-latex/unnamed-chunk-6-1.pdf}
\#\#\#\# Funcion para evaluacion de supuestos

\begin{Shaded}
\begin{Highlighting}[]
\NormalTok{ValidarSupuestos <-}\StringTok{ }\ControlFlowTok{function}\NormalTok{ (respuesta ,modelo,confianza)\{}
  
  \KeywordTok{print}\NormalTok{(}\StringTok{"En conclusion:"}\NormalTok{)}
  
  \CommentTok{# Test de normalidad}
  
\NormalTok{  shapiro <-}\StringTok{ }\KeywordTok{shapiro.test}\NormalTok{(modelo}\OperatorTok{$}\NormalTok{residuals)}
\NormalTok{  shapiro <-}\StringTok{ }\NormalTok{shapiro}\OperatorTok{$}\NormalTok{p.value}
  
\NormalTok{  lillie <-}\StringTok{ }\KeywordTok{lillie.test}\NormalTok{(modelo}\OperatorTok{$}\NormalTok{residuals)}
\NormalTok{  lillie <-}\StringTok{ }\NormalTok{lillie}\OperatorTok{$}\NormalTok{p.value}
  
    \KeywordTok{ifelse}\NormalTok{((shapiro }\OperatorTok{>}\StringTok{ }\NormalTok{confianza) }\OperatorTok{&}\StringTok{ }\NormalTok{(lillie }\OperatorTok{>}\StringTok{ }\NormalTok{confianza), }\KeywordTok{print}\NormalTok{(}\StringTok{"Existe normalidad de los errores"}\NormalTok{), }\KeywordTok{print}\NormalTok{(}\StringTok{"No existe normalidad de los errores"}\NormalTok{))}
  
  \CommentTok{# Test de homocedasticidad}
  
  
\NormalTok{  baruch <-}\StringTok{ }\KeywordTok{bptest}\NormalTok{(modelo)}
\NormalTok{  baruch <-}\StringTok{ }\NormalTok{baruch}\OperatorTok{$}\NormalTok{p.value}

\NormalTok{  golfred <-}\StringTok{ }\KeywordTok{gqtest}\NormalTok{(modelo)}
\NormalTok{  golfred <-}\StringTok{ }\NormalTok{golfred}\OperatorTok{$}\NormalTok{p.value}

  \KeywordTok{ifelse}\NormalTok{((baruch }\OperatorTok{>}\StringTok{ }\NormalTok{confianza) }\OperatorTok{&}\StringTok{ }\NormalTok{(golfred }\OperatorTok{>}\StringTok{ }\NormalTok{confianza), }\KeywordTok{print}\NormalTok{(}\StringTok{"Existe homocedasticidad"}\NormalTok{), }\KeywordTok{print}\NormalTok{(}\StringTok{"No existe homocedasticidad"}\NormalTok{))}
  
  \CommentTok{# Test de independencia}
  
\NormalTok{  indepenencia <-}\StringTok{ }\KeywordTok{dwtest}\NormalTok{(respuesta }\OperatorTok{~}\StringTok{ }\NormalTok{modelo}\OperatorTok{$}\NormalTok{residuals)}
\NormalTok{  indepenencia <-}\StringTok{ }\NormalTok{indepenencia}\OperatorTok{$}\NormalTok{p.value}
  
  \KeywordTok{ifelse}\NormalTok{((indepenencia }\OperatorTok{>}\StringTok{ }\NormalTok{confianza), }\KeywordTok{print}\NormalTok{(}\StringTok{"Hay independencia"}\NormalTok{), }\KeywordTok{print}\NormalTok{(}\StringTok{"No hay independencia"}\NormalTok{))}
  
  \CommentTok{# Construccion de la Tabla de respuestas}
  
\NormalTok{  tabla <-}\StringTok{ }\KeywordTok{rbind}\NormalTok{(shapiro, lillie, baruch, golfred, indepenencia)}
  \KeywordTok{rownames}\NormalTok{(tabla) <-}\StringTok{ }\KeywordTok{c}\NormalTok{(}\StringTok{"shapiro"}\NormalTok{, }\StringTok{"lillie"}\NormalTok{, }\StringTok{"baruch"}\NormalTok{, }\StringTok{"golfred"}\NormalTok{, }\StringTok{"independencia"}\NormalTok{)}
  \KeywordTok{colnames}\NormalTok{(tabla) <-}\StringTok{ }\KeywordTok{c}\NormalTok{(}\StringTok{"p.value"}\NormalTok{)}
  \KeywordTok{print}\NormalTok{(tabla)}

  
\NormalTok{\} }
\end{Highlighting}
\end{Shaded}

\hypertarget{se-toma-como-canidadato-el-modelo-apartir-del-r2-0.11-ajustado}{%
\paragraph{Se toma como canidadato el modelo apartir del R2 0.11
ajustado:}\label{se-toma-como-canidadato-el-modelo-apartir-del-r2-0.11-ajustado}}

STATEID = Bo + B1 * ELDERLY + B2* SCHOOLAGE

\begin{Shaded}
\begin{Highlighting}[]
\NormalTok{modelo <-}\StringTok{ }\KeywordTok{lm}\NormalTok{(STATEAID }\OperatorTok{~}\StringTok{ }\NormalTok{ELDERLY }\OperatorTok{+}\StringTok{ }\NormalTok{SCHOOLAGE)}
\KeywordTok{summary}\NormalTok{(modelo)}
\end{Highlighting}
\end{Shaded}

\begin{verbatim}
## 
## Call:
## lm(formula = STATEAID ~ ELDERLY + SCHOOLAGE)
## 
## Residuals:
##      Min       1Q   Median       3Q      Max 
## -2176.27  -473.78   -81.63   507.48  1648.95 
## 
## Coefficients:
##             Estimate Std. Error t value Pr(>|t|)    
## (Intercept)  6552.43    1578.49   4.151 0.000138 ***
## ELDERLY      -150.75      56.13  -2.685 0.009976 ** 
## SCHOOLAGE    -126.30      60.57  -2.085 0.042522 *  
## ---
## Signif. codes:  0 '***' 0.001 '**' 0.01 '*' 0.05 '.' 0.1 ' ' 1
## 
## Residual standard error: 753.8 on 47 degrees of freedom
## Multiple R-squared:   0.15,  Adjusted R-squared:  0.1138 
## F-statistic: 4.146 on 2 and 47 DF,  p-value: 0.02197
\end{verbatim}

\begin{Shaded}
\begin{Highlighting}[]
\KeywordTok{ValidarSupuestos}\NormalTok{(STATEAID,modelo,}\FloatTok{0.01}\NormalTok{)}
\end{Highlighting}
\end{Shaded}

\begin{verbatim}
## [1] "En conclusion:"
## [1] "Existe normalidad de los errores"
## [1] "Existe homocedasticidad"
## [1] "Hay independencia"
##                  p.value
## shapiro       0.47581651
## lillie        0.83183857
## baruch        0.03004474
## golfred       0.24992068
## independencia 0.45302990
\end{verbatim}

El modelo propuesto cumple todas las validaciones de supuestos y es:

STATEID = Bo + B1 * ELDERLY + B2* SCHOOLAGE

\end{document}
